\documentclass[12pt,letterpaper]{article}
\usepackage{fullpage}
\usepackage[top=2cm, bottom=4.5cm, left=2.5cm, right=2.5cm]{geometry}
\usepackage{amsmath,amsthm,amsfonts,amssymb,amscd}
\usepackage{lastpage}
\usepackage{enumerate}
\usepackage{fancyhdr}
\usepackage{mathrsfs}
\usepackage{xcolor}
\usepackage{graphicx}
\usepackage{listings}
\usepackage{hyperref}

\hypersetup{%
  colorlinks=true,
  linkcolor=blue,
  linkbordercolor={0 0 1}
}
 
\renewcommand\lstlistingname{Algorithm}
\renewcommand\lstlistlistingname{Algorithms}
\def\lstlistingautorefname{Alg.}

\lstdefinestyle{Python}{
    language        = Python,
    frame           = lines, 
    basicstyle      = \footnotesize,
    keywordstyle    = \color{blue},
    stringstyle     = \color{green},
    commentstyle    = \color{red}\ttfamily
}

\setlength{\parindent}{0.0in}
\setlength{\parskip}{0.05in}

% Edit these as appropriate
\newcommand\course{ASTR 597}
\newcommand\hwnumber{2}                  % <-- homework number
\newcommand\NetIDa{Dino}           % <-- NetID of person #1
\newcommand\NetIDb{Bektesevic}           % <-- NetID of person #2 (Comment this line out for problem sets)

\pagestyle{fancyplain}
\headheight 35pt
\lhead{\NetIDa}
\lhead{\NetIDa\\\NetIDb}                 % <-- Comment this line out for problem sets (make sure you are person #1)
\chead{\Large Summary of meter-barrier problem.}
\rhead{\course \\ \today}
\lfoot{}
\cfoot{}
\rfoot{\small\thepage}
\headsep 1.5em

\begin{document}
\textbf{Chosen papers on the topic}
\begin{itemize}
    \item Aerodynamics of solid bodies in the solar nebula, Weidenschilling 1977, MNRAS
    \item Rapid Coagulation of Porous Dust Aggregates outside the Snow Line: A Pathway to Succesful Icy Planetesimal Formation, Okuzumi et al., 2012, ApJ
    \item Streaming Instabilities in Protoplanetary Disks, Youdin and Goodman 2005, ApJ
\end{itemize}


\section*{Summary and motivation of the final project.}

The origins and creation of the Solar System has always been a topic of interest in Astronomy, reinvigorated in the last decade by missions like Deep Impact, Rosetta, Hayabusa 2 and the discovery of exoplanets. There are many different models describing the possible dynamics of the formation of a planetary system and all have had to deal with a similar underlying physical problem: the dynamics of dust in the protoplanetary disk. This dynamic has always been assumed rather simplistically, until recently at least, where planets formed by growth of orbiting dust via accretion, coagulation etc., offset by rates of fragmentation, destructive collisions and other negative-outcome scenarios in this context. Although quite logical and understandable these models have always faced several challenges - namely it is very difficult to coalesce dust, i.e. the rate of accretion is slow, i.e. the timescales at which planetesimals form are large.

Plotoplanetary disks consist mainly of gas and dust particles. In the classical picture the gas orbits around the central body partially supported by the gas pressure while the dust tends to orbit more on Keplerian-like orbits. Because of this additional supporting force exerted by pressure, gas can orbit at a lower velocities compared to appropriate expected Keplerian orbit. Since dust therefore orbits with larger velocity than gas it "plows" through the gas which causes a loss of momentum and consequently a slow drift inwards. Weidenschilling (MNRAS, 1977) examined two cases for dust-gas interaction in protoplanetary disks: 1) when the dust cross-section is smaller than the mean free path of the gas, so called Epstein regime, and 2) when the dust cross-section is larger than the mean free path of the gas, so called Stokes regime. Weidenschilling found that there is a strong correlation between the drift-speed and the particle size. He found that smaller dust particles are more strongly coupled to the gas and tend to follow its behaviour while the large particles, having larger mass and therefore being less decelerated in collisions, tend to plow through the gas unaffected. For all the particles in-between those two limits however there exists a varying drift-speed towards the central body, provided they don't coalesce or fragment on their way there. Weidenschilling finds that the largest drift speed is recovered for particles ~1m in size. But if the accretion rates, at least in the classical approach, are slow and timescales required to form a planetesimal ~1m large, but we know 1m sized objects can only live for a short period in the protoplanetary due to their large drift-speeds, how do we coalesce bodies larger than that? Such bodies would then have to be formed very quickly, or there is something wrong with how we model successful collisions. Since we can physically test accretion in the lab it is likely the former. This is the origin of the meter-barrier problem I want to cover in my final project.

Since the proposition of the problem by Weidenschilling, there have been numerous attempts to resolve the discrepancy. Okuzumi et al., (ApJ, 2012) propose a potential solution for the problem for particles beyond the snowline. They assumed a much "fluffier" composition, a lower form filing factor for the body. This enabled them to coalesce bodies above the meter-barrier for two reasons: 1) bodies with a lower form filling factor are more resistant to collisional disruption and therefore more likely to coalesce and 2) such bodies are formed by coalescing of icy material, meaning such bodies would form at larger radii from the central body where the orbital speeds are smaller leading to more successful. Such results have gained popularity after the Rosetta and Deep Impact missions both of which indicated that bodies of the Solar System might generally have a significantly smaller form filling factors than originally assumed. Unfortunately the proposed model is not able to describe how this mechanism would work for more tightly packed silicon based material closer to the central object.

Another approach involves setting up a fast accretion mechanism. Youdin and Goodman (ApJ, 2005) propose one such mechanism called streaming instability. Previously it was established that a common idea is that the gas moves at different velocity (~50m/s) than the dust particles, due to the fact gas is additionally supported by pressure, and that collisions between single dust particles and gas causes the dust to drift inwards. Instead of focusing on collisional outcomes of single dust particle and gas Youding and Goodman focused on collisional outcomes of gas and clumps of dust particles. Dust clusters increase the gas acceleration, since a more massive body colliding with a small gas molecule imparts a larger momentum onto the smaller particle, thus accelerating the orbital velocity of surrounding gas and decreasing the friction felt by the dust clump. The net result is that the drift speed of the clump is less than that of a single dust particle. Dust particles that continue to drift inwards at nominal speed catch up to the clump and increase its mass, further reducing its drift speed. Although described very corpuscularly above, these dust clumps need not be actual accreted and bound together clumps of dust particles (with respect to previous text: how would these have formed?) but local overdensities of solid matter that could have originated from some small local statistical fluctuation of density and is bound together by self-gravitational effects. 

In this project I want to cover the meter-barrier problem in more details by examining both how planetesimal composition and structure affects the accretion rates and whether there are new ideas dictating how this would occur in the inner part of the system and also examining in more details what is the theory behind the streaming instability mechanism and what are the current obstacles the idea faces.



% Weidenshilling 
%   - classic picture, but the drift inwards for 1m is to. o much, no planets formed
 
%Okuzumi et al., 2012; Kataoka et al., 2013
   %- classic pciture still might work if beyond the snowline the the planets were way fluffier than expected - Rosseta mission kind of proved that by measuring very small packing coefficient
   %- works because the radial velocities are very slow and they are very resistant to collisional disruption 
   %- but doesn't work on the inside where the aggregate is not fluffy but silicon based in composition (they bounce off of each other more easily than coalesce)
   %- so now a big ass migration mechanism is required - hard to do and doesn't make sense from composition side of things
   
%Cuzzi et al., 2001; Johansen et al., 2007
 %  - if the disk were turbulent clumps of gas might form and settle in their self-gravitational potential well 
   %- problems are that there has to be turbulence in the disk and the particles need to be larger than expected (at least in the inner disk)
   %- 2 mechanisms proposed to solve this problem: Rayleigh-Helmholtz instability or the streaming-instability.
   
%Weidenschilling, 1995 
 %  - particles cause a Rayleigh-Helmholtz instability themselves through sedimentation towards the midplane by forcing the gas to rotate faster
  % - this again depends a lot on the particle size because, smaller particles don't sediment very fast and therefore don't create a dense midplane layer where conditions for KH instability can form
   
%Youdin and Goodman, 2005 
 %  - streaming instability: In the proto-planetary nebula, hydrogen (gas) and dust (sub-cm sized fractals and ices) move at different speeds. Since the gas has a outwards pressure (friction), its speed is about 50 m/s lower than the Keplerian speed for that orbit. The dust, more massive and faster, plows through this gas and feels a headwind slowing it (friction). By reaction, the gas is pushed to accelerate (friction) with respect to the equilibrium speed given by its pressure. Random dust clusters increase the gas acceleration (more massive body colliding with a small gas molecule), decrease the friction on dust (more massive body feels less deceleration when colliding with small molecule). This slows the fall of the clump towards the center. 
   %- Other particles drifting inward at the nominal speed catch up to the clump, enhance the local over-density of solids, and further decelerate the radial drift of the clump. Thus a positive-feedback loop is established and large bodies can accreete very quickly "out of thin air". 
   

%\section*{}


\end{document}
